\setcounter{section}{0}
\setcounter{secnumdepth}{6}


% commands not part of the style
\newcommand{\mypersec}{\si{\meter\per\second}}
\newcommand{\mympersecsq}{\si{\meter\per\second\squared}}
\newcommand{\mydegpersec}{\si{\degree\per\second}}
\newcommand{\mykm}{\si{\kilo\metre}}
\newcommand{\mym}{\si{\metre}}
\newcommand{\mymm}{\si{\milli\metre}}
\newcommand{\myum}{\si{\micro\metre}}
\newcommand{\myms}{\si{\milli\second}}
\newcommand{\myus}{\si{\micro\second}}
\newcommand{\myuw}{\si{\micro\watt}}
\newcommand{\myua}{\si{\micro\ampere}}
\newcommand{\myusr}{\si{\micro\steradian}}
\newcommand{\mydegree}{\si{\degree}}
\newcommand{\mymthreedb}{$-$3~dB}
\newcommand{\Unl}[1]{\ensuremath{\underline{#1}}}

\definecolor{LightGrey}{rgb}{0.96,0.96,0.96}
\definecolor{LightGrey}{rgb}{0.95,0.95,0.95}
\definecolor{light-gray}{gray}{0.95}
\definecolor{half-gray}{gray}{0.75}
\definecolor{LightRed}{rgb}{1.0,0.9,0.9}

\newcommand{\colheightrule}{\rule[-2mm]{0mm}{6.5mm}}
\newcommand{\marginnote}[1]{ \marginpar{{\scriptsize #1}}}
\newcommand{\CJWpar}[2]{\parbox{#1}{\rule{0cm}{4mm}#2\rule[-2mm]{0cm}{4mm}}}
\newcommand{\HiLite}[1]{\fcolorbox{red}{red}{#1}}
\newcommand{\res}{\marginpar{$\ast$}}
\newcommand{\spec}[1]{\fcolorbox{half-gray}{light-gray}{#1}}

\newcommand{\TBC}[1]{%
    \par
    \noindent
    \begin{minipage}{1.0\linewidth}
      \begin{tcolorbox}[colback=purple!5, colframe=purple!95!black, coltext=black,
        coltitle=white, fonttitle=\bfseries, title={To be completed},%
        left=0mm, right=0mm, top=0mm, bottom=0mm,%
        before=\vspace{2pt}, after=\vspace{2pt}]
        % \setlength{\parskip}{0.5em} \vspace{-0.5em}
        \setlength{\parindent}{1em} \noindent
        #1
      \end{tcolorbox}
    \end{minipage}
}

\newcommand*\diff{\mathop{}\!\mathrm{d}}
\newcommand*\Diff[1]{\mathop{}\!\mathrm{d^#1}}
\newcommand*\euler{\mathop{}\!\mathrm{e}}
\newcommand*\imag{\mathop{}\!\mathrm{j}}


% set up the listings environment
%\begin{lstlisting}[hkey=value list]
%code here
%\end{lstlisting}
%\lstinline[hkey=value list]<character>source code<same character>
%\lstinputlisting[lastline=4]{listings.sty}

%\lstdefinelanguage{none}{identifierstyle=}

% set up listings package details
\lstloadlanguages{TeX,C++,XML,Python}


%\lstset{ %
\lstdefinestyle{ossimstyle}{
language=,                % choose the language of the code
upquote=true, % gives the upquote instead of the curly quote
basicstyle=\ttfamily\footnotesize,       % the size of the fonts that are used for the code
                  % alternative ((basicstyle=\ttfamily\fontsize{7}{10}\selectfont]))
numbers=none,                   % where to put the line-numbers (normally left)
numberstyle=\tiny,
%stepnumber=2,
numbersep=5pt,
showspaces=false,               % show spaces adding particular underscores
showstringspaces=false,         % underline spaces within strings
showtabs=false,                 % show tabs within strings adding particular underscores
breaklines=true,        % sets automatic line breaking
breakatwhitespace=false,    % sets if automatic breaks should only happen at whitespace
extendedchars=true,
keywordstyle=\color{red},
prebreak=\raisebox{0ex}[0ex][0ex]{$\dlsh$}, % add linebreak symbol
captionpos=b,                   % sets the caption-position to bottom
frame=none,                   % 'lines' or 'none' adds a frame around the code
%backgroundcolor=\color{LightGrey},
tabsize=2,              % sets default tabsize to 2 spaces
%escapeinside={\%}{)},          % if you want to add a comment within your code
%escapeinside={\%*}{*)},          % if you want to add LaTeX within your code
%backgroundcolor=\color{LightGrey},  % choose the background color,  add \usepackage{color}
framesep=1pt,
xleftmargin=0pt,
xrightmargin=0pt,
captionpos=t,                    % sets the caption-position to top
%deletekeywords={...},            % if you want to delete keywords from the given language
numberbychapter=false,
}

\DeclareCaptionFont{white}{\color{white}}
\DeclareCaptionFormat{listing}{\colorbox[cmyk]{0.43, 0.35, 0.35,0.01}{\parbox{\textwidth}{\hspace{15pt}#1#2#3}}}
\captionsetup[lstlisting]{format=listing,labelfont=white,textfont=white, singlelinecheck=false, margin=0pt, font={bf,footnotesize}}

% make inline listing larger than display
%https://latex.org/forum/viewtopic.php?t=2072
\makeatletter
\lst@AddToHook{TextStyle}{\let\lst@basicstyle\ttfamily\footnotesize\fontfamily{pcr}\selectfont}
\makeatother

\lstset{style=ossimstyle}


\lstdefinestyle{pythonstyle}{
  backgroundcolor=\color{LightGrey},  % choose the background colour,  add \usepackage{color}
  language=Python,
}

\lstdefinestyle{cppstyle}{
  backgroundcolor=\color{LightGrey},  % choose the background colour,  add \usepackage{color}
  language=C++,
}

\lstdefinestyle{xmlstyle}{
  backgroundcolor=\color{LightGrey},  % choose the background colour,  add \usepackage{color}
  language=XML,
}

\lstdefinestyle{nonestyle}{
  backgroundcolor=\color{LightGrey},  % choose the background colour,  add \usepackage{color}
}

\renewcommand{\arraystretch}{1.5}

\newcommand{\myhline}{\ \hrulefill \ }

%\minipsize{90mm}{some text}
\newcommand{\minipsize}[2]{\begin{minipage}[t]{#1}\begin{flushleft}#2\end{flushleft}\end{minipage}}


%%%%%%%%%%%%%%%%%%%%%%%%%%%%%%%%%%%%%%%%%%%%%%%%%%%%%%%%%%%%%
%\usepackage{etoolbox}
%\makeatletter
%\patchcmd{\thebibliography}{%
%  \chapter{\bibname}\@mkboth{\MakeUppercase\bibname}{\MakeUppercase\bibname}}{%
%  \chapter{References}}{}{}
%\makeatother

%\bibliographystyle{plain}
%\bibliographystyle{acm}
\bibliographystyle{bib/dpsscjw}

\newcommand{\tabrule}{\rule[-2mm]{0mm}{6mm}}

%% Symbols used in missile dynamics
%% See ../Missile Airframe-Guidance Models/c03_01.tex for definitions
\newcommand{\fax}{\ensuremath{f_\mathrm{ax}}}
\newcommand{\fay}{\ensuremath{f_\mathrm{ay}}}
\newcommand{\faz}{\ensuremath{f_\mathrm{az}}}
\newcommand{\fth}{\ensuremath{f_\mathrm{th}}}
\newcommand{\ribi}{\ensuremath{r_{ib}^i}}
\newcommand{\vibi}{\ensuremath{v_{ib}^i}}
\newcommand{\cib}{\ensuremath{C_i^b}}
\newcommand{\wibb}{\ensuremath{\omega_{ib}^b}}
\newcommand{\Wibb}{\ensuremath{\Omega_{ib}^b}}
\newcommand{\qdyn}{\ensuremath{q_\mathrm{dyn}}}
\newcommand{\sref}{\ensuremath{S_\mathrm{ref}}}
\newcommand{\vsound}{\ensuremath{v_\mathrm{s}}}
\newcommand{\nbt}{\ensuremath{\hat{t}}}
\newcommand{\tburn}{\ensuremath{t_\mathrm{burn}}}
\newcommand{\tign}{\ensuremath{t_\mathrm{ign}}}
\newcommand{\alphamax}{\ensuremath{\alpha_\mathrm{max}}}
\newcommand{\cazero}{\ensuremath{C_{A_0}}}
\newcommand{\cabase}{\ensuremath{C_{A_\mathrm{base}}}}
\newcommand{\cainc}{\ensuremath{C_{A_i}}}
\newcommand{\caincfull}{\ensuremath{C_{A_{if}}}}
\newcommand{\caincempty}{\ensuremath{C_{A_{ie}}}}
\newcommand{\cn}{\ensuremath{C_N}}
\newcommand{\cnfull}{\ensuremath{C_{N_f}}}
\newcommand{\cnempty}{\ensuremath{C_{N_e}}}
\newcommand{\xref}{\ensuremath{X_\mathrm{ref}}}
\newcommand{\czalpha}{\ensuremath{{C_{Z_\alpha}}}}
\newcommand{\chk}{$\times$}

